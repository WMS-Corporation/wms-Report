\section{Design}
\subsection{Metodologie di Sviluppo}

Il team ha adottato un approccio ispirato a \textbf{SCRUM} per il processo di sviluppo,
che comprende la creazione iniziale del backlog del prodotto e la pianificazione
del primo sprint durante un incontro iniziale.
Gli sprint sono stati strutturati con una durata settimanale al fine di ottenere
risultati concreti e di valore per gli stakeholder. Sono stati condotti frequenti
incontri all'inizio e alla fine di ciascuno sprint, mantenendo un backlog dello sprint
per tenere traccia dell'organizzazione del lavoro.

Il primo sprint si è concentrato sull'analisi del settore, inclusa la conduzione di
interviste simulate con il committente e la creazione di tutti gli artefatti necessari
per l'analisi del dominio. Inoltre, in questo primo sprint abbiamo definito gli strumenti
DevOps e impostato l'infrastruttura per l'automazione della build, l'integrazione continua
e il controllo di versione, per poterne beneficiare fin da subito. Questo sprint iniziale
è stato di fondamentale importanza per il progetto ed è stato svolto in modo collaborativo
da tutti i membri del gruppo.

Successivamente, il team ha deciso di suddividere il lavoro in quattro sprint aggiuntivi:

\begin{itemize}
    \item \textbf{Sprint 2-3}: dedicati alla creazione e testing dei microservizi (con relativa gestione della consistenza dei dati attraverso MongoDB) per ciascun bounded context che era stato identificato con un focus sulle relative API RestFul;
    \item \textbf{Sprint 4-5}: dedicati alla creazione e testing di una webApp per la gestione dell’interfaccia utente;
    \item \textbf{Sprint 6-7}: dedicati alla rimodellazione e ristrutturazione del codice Client con i relativi test nonché alla stesura stessa di questa relazione.
\end{itemize}

In totale, sono stati completati sette sprint nel corso del processo di sviluppo del progetto.

Parallelamente al processo di sviluppo, è stata adottata una metodologia di design chiamata \textbf{UCD (User Centered Design)}. La scelta di questa metodologia è stata determinata dal fatto che, fin dalle prime fasi di progettazione, si è voluto dare priorità alla \textbf{HCI (Human Computer Interaction)}, concentrando l’attenzione sulle necessità degli utenti e sull’obiettivo di garantire un’usabilità ottimale per migliorare l’esperienza dell’utente. Per quanto riguarda l’implementazione dell’applicazione, il focus si è posto sul dover soddisfare le esigenze e le richieste degli utenti. Gli User a cui si farà riferimento successivamente sono rappresentati attraverso una serie di profili chiamati “personas”, che riflettono un possibile pubblico di riferimento dell’applicazione.
\newpage
\subsection{Target User Analysis}

Nel contesto della gestione di un magazzino alimentare, sia gli utenti amministrativi che operativi si trovano a dover affrontare una serie di compiti complessi e cruciali. Gli utenti amministrativi sono responsabili della supervisione generale delle operazioni, del monitoraggio delle scorte, della gestione degli ordini e della garanzia della qualità e sicurezza degli articoli. D’altra parte, gli utenti operativi sono responsabili dell’esecuzione delle attività quotidiane, come il carico e lo scarico delle merci e il mantenimento della pulizia e dell’ordine nel magazzino.

\subsubsection{Descrizione del Target di Utenti}

L'applicazione che intendiamo sviluppare fornirà funzionalità specifiche per entrambi i tipi di utenti. Per gli utenti amministrativi, l’applicazione offrirà una dashboard centralizzata che fornirà informazioni dettagliate sullo stato degli ordini, dei task e il monitoraggio delle temperature nelle zone refrigerate. Inoltre, attraverso notifiche proattive, gli utenti amministrativi saranno avvisati tempestivamente di situazioni critiche come deviazioni di temperatura o scorte a rischio di esaurimento. Per gli utenti operativi, l’applicazione semplificherà notevolmente le operazioni quotidiane. Una funzionalità chiave sarà la possibilità di eseguire facilmente compiti come il carico e lo scarico delle merci attraverso un’interfaccia intuitiva e user-friendly. Inoltre, l’applicazione fornirà agli utenti operativi istruzioni dettagliate e supporto in tempo reale per garantire l’esecuzione corretta delle attività.

\subsubsection{Analisi Strategica per la Realizzazione dell'Applicativo}

Per soddisfare pienamente i requisiti degli utenti, sono state effettuate delle analisi strategiche:

\begin{itemize}
    \item \textbf{Conoscenza dei Compiti Utente}: Comprendere le procedure necessarie per la gestione di un magazzino alimentare, inclusi i passaggi che un utente amministrativo deve compiere per monitorare e gestire le scorte e gli ordini, e le azioni che un utente operativo deve eseguire quotidianamente.
    \item \textbf{Ottimizzazione dell’Applicativo}: Il servizio deve essere intuitivo, poiché gli utenti operativi e amministrativi potrebbero non avere familiarità con strumenti digitali avanzati. La semplicità e l'intuitività nell'uso delle varie funzionalità renderanno l'uso del servizio più appagante.
    \item \textbf{Affidabilità e Prestazioni}: Gli utenti si aspettano che l'applicativo funzioni come supporto affidabile per la gestione del magazzino. Di conseguenza, l’applicazione non deve presentare anomalie progettuali o funzionali che possano compromettere l’efficienza operativa.
\end{itemize}

\subsubsection{Categorie di Utenti}

Il target di utenza è suddiviso in due categorie principali:

\begin{itemize}
    \item \textbf{Utenti Amministrativi}: Responsabili della supervisione generale del magazzino, monitoraggio delle scorte, gestione degli ordini e garanzia della qualità e sicurezza degli articoli. Hanno bisogno di strumenti di gestione avanzati, dashboard centralizzate e notifiche proattive per gestire situazioni critiche.
    \item \textbf{Utenti Operativi}: Responsabili delle operazioni quotidiane nel magazzino, come carico e scarico delle merci e mantenimento dell'ordine. Necessitano di un’interfaccia user-friendly, istruzioni dettagliate e supporto in tempo reale per eseguire le loro attività in modo efficiente.
\end{itemize}


\subsubsection{Personas}

Di seguito vengono riportati diversi personas, rappresentazioni ipotetiche di utenti che devono utilizzare l’applicativo per assolvere determinati compiti.

\textbf{Persona: Mario, 45 anni, Responsabile del Magazzino}

\textbf{Scenario d’uso}: Mario è il responsabile del magazzino alimentare. Ha un'esperienza di oltre 20 anni nel settore e si occupa principalmente della supervisione delle operazioni, del monitoraggio delle scorte e della gestione degli ordini.

\begin{itemize}
    \item Mario accede al sito effettuando il login con le sue credenziali amministrative.
    \item Utilizza la dashboard per monitorare lo stato del magazzino, inclusi lo stato degli ordini, task e temperatura delle varie zone.
    \item Riceve notifiche proattive riguardanti deviazioni di temperatura e scorte a rischio di esaurimento.
    \item Gestisce gli ordini e monitora la qualità degli articoli stoccati.
\end{itemize}

\textbf{Persona: Luca, 30 anni, Operatore di Magazzino}

\textbf{Scenario d’uso}: Luca è un operatore di magazzino che si occupa del carico e dello scarico delle merci. È esperto nell'uso di attrezzature per la movimentazione dei materiali e segue rigorosamente le procedure di sicurezza.

\begin{itemize}
    \item Luca accede al sito utilizzando le sue credenziali operative.
    \item Utilizza un’interfaccia intuitiva per registrare il carico e lo scarico delle merci.
    \item Segue istruzioni dettagliate e riceve supporto in tempo reale per garantire l’esecuzione corretta delle attività.
    \item Mantiene la pulizia e l’ordine nel magazzino seguendo le procedure indicate nell’applicazione.
\end{itemize}

Questi profili aiutano a mantenere il focus sulle esigenze degli utenti durante tutte le fasi di sviluppo e progettazione dell'applicazione. La metodologia \textbf{UCD (User Centered Design)} e il principio \textbf{KISS (Keep It Simple, Stupid)} sono stati adottati per garantire che l'applicazione sia facile da usare e risponda efficacemente alle esigenze degli utenti, migliorando l'efficienza operativa e la soddisfazione generale.


