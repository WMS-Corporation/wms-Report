\section{Test}
\subsection{Euristiche di Nielsen}
\subsection{Test di usabilità}
Oltre alle verifiche di usabilità tramite euristiche, sono stati raccolti feedback iterativi da un gruppo di utenti.\\
Sono stati assegnati i due differenti ruoli di amministratore e operativo al gruppo di utenti, ed è stato richiesto
loro di svolgere determinati compiti senza suggerimenti o consigli.\\

\begin{itemize}
    \item \textbf{Utenti Amministrativi}: Agli utenti amministrativi è stato chiesto di navigare attraverso la dashboard centralizzata per monitorare i movimenti degli articoli, lo stato degli ordini e le temperature nelle zone refrigerate.\\ Inoltre, è stato richiesto loro di gestire le notifiche proattive, affrontando situazioni critiche come deviazioni di temperatura o scorte a rischio di esaurimento.\\ Infine, hanno dovuto eseguire compiti come l'aggiunta, modifica ed eliminazione di prodotti e ordini.

    \item \textbf{Utenti Operativi}: Agli utenti operativi è stato chiesto di eseguire attività quotidiane come il carico e lo scarico delle merci utilizzando l'interfaccia intuitiva dell'applicazione.\\ Dovevano seguire le istruzioni dettagliate fornite in tempo reale per garantire l'esecuzione corretta delle attività.\\ Inoltre, gli è stato richiesto di completare i task assegnati e di segnalarne il completamento tramite l'icona \textbf{Done}.
\end{itemize}

Questo approccio ha permesso di raccogliere feedback preziosi da parte degli utenti, identificando aree di
miglioramento e garantendo che l'applicazione risponda efficacemente alle esigenze sia degli utenti amministrativi
che operativi nella gestione del magazzino alimentare.
\subsection{Unit test}