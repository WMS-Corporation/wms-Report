\newpage
\section{Conclusioni}

L’obiettivo del progetto era quello di creare un’applicazione web completa e intuitiva per la gestione e il monitoraggio di un magazzino alimentare.
\\In particolare, il sistema è progettato per soddisfare le esigenze di due tipi di utenti distinti: gli operatori di
magazzino (Operational) e il personale amministrativo (Administrator).\\ Gli operatori di magazzino sono responsabili
delle attività quotidiane del magazzino, compreso il carico e lo scarico della merce, mentre il personale amministrativo
ha compiti più orientati alla supervisione e alla gestione generale del sistema.\\

Attraverso un'analisi dettagliata delle esigenze e dei requisiti del contesto operativo, siamo stati in grado di
progettare e sviluppare un'applicazione partendo da zero, adottando il pattern architetturale Domain-Driven Design (DDD).\\
Questo approccio ci ha permesso di creare una soluzione su misura, completamente adattata alle specifiche necessità di un magazzino alimentare.\\

Lo sviluppo del progetto ci ha consentito di migliorare significativamente le nostre capacità di lavoro in team e di sviluppo collaborativo.\\
Utilizzando lo stack tecnologico MERN, abbiamo avuto l'opportunità di interfacciarci con tecnologie largamente utilizzate nel contesto lavorativo
e altamente versatili nella realizzazione di progetti di vario tipo.\\

Per quanto riguarda le tecnologie web, il framework front-end React si è rivelato facilmente comprensibile e applicabile,
grazie ai laboratori svolti durante l'anno. Nella parte di backend, abbiamo implementato una struttura appresa e successivamente
adattata durante le lezioni, permettendoci di sviluppare un applicativo solido con un'architettura ampiamente scalabile.\\

L'adozione di una metodologia ispirata a SCRUM ha svolto un ruolo fondamentale nel guidare il processo di sviluppo,
consentendo una gestione efficace dei compiti e una distribuzione equa delle responsabilità tra i membri del team.\\Grazie
all'implementazione delle pratiche di Continuous Integration, siamo stati in grado di individuare e risolvere tempestivamente
eventuali problemi, assicurando un ciclo di sviluppo efficiente e una costante evoluzione dell'applicazione.\\
Inoltre, questo progetto ci ha offerto l'opportunità di ampliare le nostre competenze e conoscenze nel campo dello sviluppo software.\\
Sviluppando l'applicazione da zero, abbiamo potuto approfondire la nostra comprensione dei principi fondamentali della progettazione
e dello sviluppo software, acquisendo competenze preziose che ci saranno utili in progetti futuri.\\
In conclusione, il completamento di questo progetto rappresenta un importante traguardo nel nostro percorso professionale,
arricchendoci di nuove competenze e fornendoci una solida base per affrontare le sfide future nel campo dello sviluppo software.