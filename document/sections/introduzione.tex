\section{Introduzione}

Il progetto mira alla realizzazione di un'applicazione web completa per la gestione e il monitoraggio di un magazzino alimentare.\\
L'applicazione sarà progettata per permettere agli utenti con mansioni amministrative di gestire e monitorare il
magazzino, mentre agli utenti operativi faciliterà compiti pratici come la movimentazione delle merci, inclusi carico e scarico.

\subsection{Scenario}

Nel contesto della gestione di un magazzino alimentare, sia gli utenti amministrativi che operativi si trovano a dover
affrontare una serie di compiti complessi e cruciali.\\ Gli utenti amministrativi sono responsabili della supervisione
generale delle operazioni, del monitoraggio delle scorte, della gestione degli ordini e della garanzia della qualità e
sicurezza degli articoli.\\ Dall' parte, gli utenti operativi sono responsabili dell'esecuzione delle attività
quotidiane, come il carico e lo scarico delle merci.\\
L'applicazione che intendiamo sviluppare fornirà funzionalità specifiche per entrambi i tipi di utenti.\\
Per gli utenti amministrativi, l'applicazione offrirà una dashboard centralizzata che fornirà informazioni dettagliate
sui movimenti degli articoli, lo stato degli ordini e il monitoraggio delle temperature nelle zone refrigerate.\\
Inoltre, attraverso notifiche proattive, gli utenti amministrativi saranno avvisati tempestivamente di eventi importanti
come deviazioni di temperatura in situazioni critiche o scorte a rischio di esaurimento.\\
Per gli utenti operativi, l'applicazione semplificherà notevolmente le operazioni quotidiane.\\
Una funzionalità chiave sarà la possibilità di eseguire facilmente compiti come il carico e lo scarico delle
merci attraverso un'interfaccia intuitiva e user-friendly.\\
Inoltre, l'applicazione fornirà agli utenti operativi istruzioni dettagliate e supporto in tempo reale per garantire l'esecuzione corretta delle attività.\\
Lo scopo di questo progetto è realizzare un'applicazione per la gestione e il monitoraggio di un magazzino con zone
refrigerate, attraverso l'adozione di un approccio basato sul Domain-Driven Design (DDD).\\
Questo significa che ci siamo concentrati sulla comprensione approfondita e sulla modellazione accurata del dominio di business nella fase iniziale di analisi.\\
L'approccio DDD è stato integrato con strumenti per la Continuous Integration e il DevOps, che ci hanno guidato verso le seguenti scelte di sviluppo:
\begin{itemize}
    \item Scomposizione dell’architettura monolitica del Server in più servizi indipendenti;
    \item Utilizzo di tecnologie più recenti e performanti;
    \item Utilizzo dei principi di progettazione del software DRY, KISS e SOLID;
    \item Aggiunta di test per garantire la correttezza e aumentare la qualità del codice.
\end{itemize}
\newpage
\subsection{Descrizione in dettaglio}

Il progetto propone lo sviluppo di una soluzione software web-based per la gestione e il monitoraggio di un magazzino
alimentare, rivolgendosi sia agli utenti amministrativi (\textbf{Admin User}) che agli utenti operativi (\textbf{Operational User}).\\
L'obiettivo principale è fornire agli utenti amministrativi strumenti per gestire e monitorare il magazzino,
mentre agli utenti operativi si intende facilitare le operazioni pratiche come la movimentazione delle merci, inclusi carico e scarico.\\
La web application si articola in due interfacce principali, una dedicata agli utenti amministrativi e una agli utenti operativi.\\
L'accesso a entrambe le interfacce è consentito solo previa registrazione e autenticazione.\\
All'interno dell'applicazione, sarà possibile:
\begin{itemize}
    \item per gli utenti amministrativi: monitorare le scorte, gestire gli ordini, assegnare task specifici agli operai, controllare movimentazioni degli articoli e stato delle zone refrigerate, gestire gli utenti.
    \item per gli operai: consultare ed eventualmente modifiare (avviare, completare) i compiti a loro assegnati.
\end{itemize}

Il sistema sarà caratterizzato da due componenti:
\begin{itemize}
    \item \textbf{Web Application}: accessibile da qualunque dispositivo connesso alla rete aziendale, fornisce un'interfaccia user-friendly per interagire con il sistema.
    \item \textbf{Web Server}: composto da microservizi, gestisce la logica di business e le operazioni di backend dell'applicazione.
\end{itemize}
\newpage