\section{Requisiti}

In questo capitolo, esploreremo i requisiti necessari per garantire il successo del progetto.
Le varie sezioni di seguito delineeranno in dettaglio le diverse tipologie di requisiti.

\begin{itemize}
    \item \textbf{Requisiti di Business}: questi requisiti delineano le caratteristiche chiave che il sistema deve possedere per essere considerato adeguato alle esigenze aziendali.
    \item \textbf{Requisiti Utente}: questi requisiti esprimono le necessità e le azioni che gli utenti devono poter compiere interagendo con il sistema.
    \item \textbf{Requisiti Funzionali}: questi requisiti riguardano le funzionalità specifiche che il sistema deve offrire agli utenti. Vengono definiti in base ai requisiti utente identificati in precedenza.
    \item \textbf{Requisiti non Funzionali}: questi requisiti si concentrano su aspetti come prestazioni, sicurezza e usabilità del sistema, che non sono strettamente legati alle funzionalità ma sono essenziali per garantirne il corretto funzionamento.
    \item \textbf{Requisiti di Implementazione}: questi requisiti influenzano l'intero processo di sviluppo del sistema, ad esempio stabilendo l'uso di determinati linguaggi di programmazione o strumenti software.
\end{itemize}

\subsection{Requisiti Utente}

I requisiti utente esprimono le necessità e le azioni che gli utenti devono
poter compiere interagendo con il sistema.

Quando si tratta di accedere al sistema, sia gli utenti amministrativi che gli utenti
operativi devono essere in grado di autenticarsi senza intoppi. Questo permetterà loro di
accedere alle funzionalità vitali dell'applicazione e svolgere i propri compiti in modo efficiente.

\subsubsection{Operational}

\begin{itemize}
    \item Aggiungere nuovi articoli al magazzino quando arrivano forniture.
    \item Rimuovere articoli esistenti dal magazzino quando vengono spediti ai clienti o trasferiti altrove.
\end{itemize}

\subsubsection{Administrator}

\begin{itemize}
    \item Tracciare ogni movimento degli articoli all'interno del magazzino per garantire una sincronizzazione accurata dell'inventario.
    \item Effettuare ordini quando il livello di inventario lo richiede per mantenere un adeguato livello di scorte.
    \item Monitorare attentamente la temperatura nelle zone refrigerate per prevenire la perdita di merce dovuta a condizioni non ottimali.
    \item Assegnare task giornalieri ai dipendenti operativi per mantenere un flusso di lavoro ordinato e fluido.
    \item Generare report periodici sullo stato del magazzino per l'ottimizzazione dei processi e la valutazione delle performance.
    \item Gestire gli utenti modificando il loro ruolo e, in generale, i loro dati se necessario.
\end{itemize}

\subsection{Requisiti Funzionali}

I requisiti funzionali riguardano le funzionalità specifiche che il sistema
deve offrire agli utenti. Vengono definiti in base ai requisiti utente identificati in precedenza.

In particolare il sistema deve consentire agli utenti di:
\begin{itemize}
    \item Registrarsi o effettuare il login all’apertura del sito.
    \item Visualizzare, aggiungere, modificare o eliminare risorse.
    \item Poter monitorare informazioni sulle risorse (quantità, posizione, scadenza, temperatura cella per risorse refrigerate).
    \item Visualizzare i task assegnati agli operatori di magazzino da parte del personale amministrativo, includendo dettagli sulle attività da svolgere e le relative scadenze.
    \item Gestire gli account degli utenti operativi, consentendo al personale amministrativo di modificare le credenziali di accesso e rimuovere gli account non più necessari.
    \item Gestire e assegnare task agli operatori di magazzino, permettendo al personale amministrativo di definire le attività, specificarne la priorità e stabilirne le scadenze.
    \item Gestire gli ordini della merce, consentendo al personale amministrativo di creare, modificare, approvare o cancellare gli ordini.
    \item Gestire la logistica e i prodotti del magazzino, permettendo al personale amministrativo di registrare le movimentazioni di merce, assegnare posizioni nel magazzino e categorizzare i prodotti.
\end{itemize}

\subsection{Requisiti Non Funzionali}

I requisiti non funzionali si concentrano su aspetti come prestazioni,
sicurezza e usabilità del sistema, che non sono strettamente legati alle funzionalità ma sono
essenziali per garantirne il corretto funzionamento.

Nel dettaglio:
\begin{itemize}
    \item \textbf{Usabilità}: garantire che il sistema sia intuitivo e facile da usare per gli utenti, riducendo al minimo la necessità di formazione aggiuntiva.
    \item \textbf{Sicurezza dei Dati}: implementare misure di autenticazione, autorizzazione e crittografia per proteggere le informazioni sensibili degli utenti e prevenire accessi non autorizzati.
    \item \textbf{Reattività}: assicurare che il sistema risponda prontamente alle richieste degli utenti, evitando ritardi e garantendo un'esperienza d'uso fluida.
    \item \textbf{Responsività}: garantire che l'applicazione sia adattabile a diverse dimensioni di schermo e dispositivi, consentendo agli utenti di accedere e utilizzare il sistema da qualsiasi dispositivo.
    \item \textbf{Affidabilità}: garantire che il sistema sia disponibile in modo continuo, riducendo al minimo i tempi di inattività e garantendo che gli utenti possano accedere alle funzionalità in qualsiasi momento.
\end{itemize}

\subsection{Requisiti di Business}

I requisiti di business delineano le caratteristiche chiave che il sistema deve possedere
per essere considerato adeguato alle esigenze aziendali.

I requisiti di business previsti per WMS sono:
\begin{itemize}
    \item \textbf{Gestione Semplificata dell'Inventario}:
    \begin{itemize}
        \item Fornire un'interfaccia intuitiva per monitorare e gestire l'inventario del magazzino.
        \item Garantire una visualizzazione chiara delle risorse disponibili.
    \end{itemize}
    \item \textbf{Controllo Automatico della Temperatura}:
    \begin{itemize}
        \item Implementare un sistema per il monitoraggio e la regolazione automatica della temperatura.
        \item Assicurare le condizioni ottimali per la conservazione dei prodotti.
    \end{itemize}
    \item \textbf{Accesso Facile per Utenti Operativi e Amministrativi}:
    \begin{itemize}
        \item Creare account distinti per gli utenti operativi e amministrativi.
        \item Consentire agli utenti operativi di visualizzare in modo semplice ed intuitivo i task assegnati dal personale amministrativo.
        \item Consentire agli utenti amministrativi di monitorare e gestire efficacemente le risorse del magazzino, compresa la supervisione delle attività operative e la generazione di report dettagliati.
    \end{itemize}
\end{itemize}

\subsection{Requisiti Implementativi}

I requisiti implementativi influenzano l'intero processo di sviluppo del
sistema, ad esempio stabilendo l'uso di determinati linguaggi di programmazione o strumenti software.

\begin{itemize}
    \item L'applicazione Web e la componente di gestione della logica saranno sviluppate usando lo stack \textbf{MERN}.
    \item Il testing del sistema sarà effettuato utilizzando il framework di test \textbf{Jest}.
    \item Per la persistenza dei dati verrà utilizzato un database di tipo non-relazionale, \textbf{MongoDB}.
    \item L'applicazione deve garantire continuità di servizio, mostrando un messaggio di errore all'utente in caso di interruzione.
\end{itemize}